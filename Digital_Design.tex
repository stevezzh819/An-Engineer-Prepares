\subsection{Digital Design and computer architecture}
\subsubsection{Number Systems}
\textbf{Binary-Coded Decimal}
\begin{enumerate}
    \item BCD is a code to represent ten decimal digits (0-9)
    \item Each decimal digit is reprensented by a 4-bit binary number
    \item Six number are not used in BDC
        \begin{itemize}
            \item Any addition more than 9 need to be plus a 6 $(0110)_{B}$, and the carry is 1
        \end{itemize}
    
\end{enumerate}
\subsubsection{Boolean Algebra}
\begin{enumerate}
    \item Distributive Law
    \[A + (B \cdot C)=(A+B)\cdot(A+C)\]
    \item Consensus Law
    \[AB + \bar{A}C+BC=AB+\bar{A}C\]
    \item Logical Adjacency
    \[AB+A\bar{B}=A\]
    \item Absorption Law
    \[A+A\cdot B=A\]
    \[A+\bar{A} \cdot B=A+B\]
    \item De Morgan's Law
    \[\overline{A+B}=\bar{A} \cdot \bar{B}\]
    \item Simplifying Using Boolean algebra laws
    \begin{itemize}
        \item Using Absorption Law, Logical Adjacency and Consensus Law sequencialy to obtained the simplest expression
    \end{itemize}
\end{enumerate}

\subsubsection{SOP and POS}
\begin{enumerate}
    \item \textbf{Minterm} is a product term that contains all variables in the function
    \item \textbf{Maxterm} is a sum term that contains all variables in the function
    \item \textbf{Canonical Form} is term expressed either by CSOP or CPOS
    \item SOP $\rightarrow$ Sum of Products
    \begin{itemize}
        \item If any product in SOP is \textbf{1}, the function is \textbf{1}. Otherwise is \textbf{0}
        \item CSOP only includes the terms that outcome is \textbf{1}
    \end{itemize}
    \item POS $\rightarrow$ Product of Sum
        \begin{itemize}
            \item If any sum in POS is \textbf{0}, the function is \textbf{0}. Otherwise is \textbf{1}
            \item CPOS only includes the terms that outcome is \textbf{0}
        \end{itemize}
\end{enumerate}

\subsubsection{Postive and Negative Logic}
If a logic is said to be positive or negative, they should follow the following rules
\begin{table}[h]
    \centering
    \begin{tabular}{|c|c|c|}
        \hline
        Voltage level & Positive logic value& Negative logic value \\
        \hline
        H & 1 & 0\\
        \hline
        L & 0 & 1\\
        \hline
    \end{tabular}
\end{table}

\subsubsection{Sequential Circuits}
\textbf{Synchronous and Asynchronous} \\
A \textbf{Sequential} logic circuits is said to be Synchronous is responded to inputs at discrete time instants governed by a clock input. Synchronous logic circuits need a clock. \\
An \textbf{Asynchronous} circuits responds whenever input signals change. No need clock\\
\\
\textbf{SR Flip-flop} \\
SR latch is the simplest circuit that contains memory, S is stand for set and R is stand for reset.
\begin{table}[h]
    \centering
    \begin{tabular}{|c|c|c|}
    \hline
         S & R & Output  \\
         \hline
         0 & 0 & \textcolor{red}{Hold} \\
         \hline
         0 & 1 & Q=0 \\
         \hline
         1 & 0 & Q=1 \\
         \hline
    \end{tabular}
    \caption{Truth table for SR FF}
\end{table}

\subsubsection{Linear Feedback Shift Register}

