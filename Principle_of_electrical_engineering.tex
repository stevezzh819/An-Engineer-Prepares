\subsection{Principle of Electrical engineering}
\subsubsection{Electronic Components}
\begin{enumerate}
    \item \textbf{Capacitance} is a parameter that quantifies the amount of charge required to make one volt p.d across the plates.
\[C=\frac{q}{V}\]
The e field created by the separate of charge is:
\[E=\frac{q}{\epsilon A}\] 
\[\epsilon_0 =8.85\times10^12\ F/m\]
where $\epsilon = \epsilon_r \epsilon_0$, and $\epsilon_r$ is the relative permittivity of a given material. 
\[V=E\times d=\frac{qd}{\epsilon A}\]
where $d$ is the separation of the two plates, $\epsilon$ is the permittivity or dielectric constant.\\
Capacitance is :
\[C=\frac{q}{V}=\frac{\epsilon A}{d}\]
\[i_C = C\frac{dv_C}{dt}\]
Voltage in terms of Current:
\[q(t)=\int^t_{t_0}{i(t)\, dt+q(t_0)}\]
\[v(t)=\frac{1}{C}\int^t_{t_0}{i(t)\, dt+v(t_0)}\]
Stored Energy:
\[w(t)=\frac{1}{2}Cv^2(t)\]
\[w(t)=\frac{1}{2}v(t)q(t)\]
\[w(t)=\frac{q^2(t)}{2C}\]
Capacitance in parallel:
\[C_{eq} = C_1+C_2+C_3\]
Capacitance in series:
\[C_{eq}=\frac{1}{\frac{1}{C_1}+\frac{1}{C_2}+\frac{1}{C_3}}\]
Capacitor Voltage while being discharged:
\[v_{C}(t)=V_0e^{-\frac{t}{RC}}\]
The speed of discharging is determined by the product of R and C, this is known as the \textbf{time constant}: $\tau=RC$ for a RC circuit.\\
Usually it takes 5$\tau$ to fully charge or discharge a capacitor.\\
Capacitor Voltage while charging:
\[v_{C}(t)=V_S(1-e^{-\frac{t}{\tau}})\] when charging from 0 $V$.\\
Current discharged in Capacitor:
\[i_C(t)=\frac{V_s}{R}e^{-\frac{t}{RC}}\]
    \item \textbf{Inductor} stores energy in a circuit and the working principle of an inductor is based on magnetic effect of electric current.\\
\textbf{Magnetic Flux density} is the amount of magnetic flux passing through per unit area $B = \frac{\phi}{A}$, where, $\phi$ is the magnetic flux and $A$ is the area which flux passes.
Flux-linkage of a coil:
\[\Psi = N\phi\]
\[\Psi = Li\]
The constant proportionality of L is called \textbf{inductance}. The unit of inductance is Henry($H$). \\
\textit{Faraday's Law}, the magnitude of the induced voltage is equal to the rate of change of flux-linkage of the coil.
\[v_L=\frac{d\Psi}{dt}\]
Element law of inductor:
\begin{align*}
v_L &= \frac{dLi}{dt}\\
    &= L\frac{di}{dt}
\end{align*}
Current and voltage relationship in inductor:
\[i(t)=\frac{1}{L}\int^t_{t_0}v(t)dt+i(t_0)\]
instantaneous power of an inductor:
\[p_L =v_Li_L =L\frac{di_L}{dt}i_L\]
Stored Energy:
\[w(t)=L\int^{I}_{0}i_Ld_{iL} = \frac{1}{2}Li^2(t)\]
Inductors in parallel:
\[L_{eq}=\frac{1}{\frac{1}{L_1}+\frac{1}{L_2}+\frac{1}{L_e}}\]
Inductors in series:
\[L_{eq}=L_1+L_2+L_3\]
\end{enumerate}